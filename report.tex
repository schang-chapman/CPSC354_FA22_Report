\documentclass{article}

\usepackage{amsthm}
\usepackage{amsfonts}
\usepackage{amsmath}
\usepackage{amssymb}
\usepackage{fullpage}
\usepackage[usenames]{color}
\usepackage{hyperref}
  \hypersetup{
    colorlinks = true,
    urlcolor = blue,       % color of external links using \href
    linkcolor= blue,       % color of internal links 
    citecolor= blue,       % color of links to bibliography
    filecolor= blue,        % color of file links
    }
    
\usepackage{listings}

\definecolor{dkgreen}{rgb}{0,0.6,0}
\definecolor{gray}{rgb}{0.5,0.5,0.5}
\definecolor{mauve}{rgb}{0.58,0,0.82}

\lstset{frame=tb,
  language=haskell,
  aboveskip=3mm,
  belowskip=3mm,
  showstringspaces=false,
  columns=flexible,
  basicstyle={\small\ttfamily},
  numbers=none,
  numberstyle=\tiny\color{gray},
  keywordstyle=\color{blue},
  commentstyle=\color{dkgreen},
  stringstyle=\color{mauve},
  breaklines=true,
  breakatwhitespace=true,
  tabsize=3
}

\theoremstyle{theorem} 
   \newtheorem{theorem}{Theorem}[section]
   \newtheorem{corollary}[theorem]{Corollary}
   \newtheorem{lemma}[theorem]{Lemma}
   \newtheorem{proposition}[theorem]{Proposition}
\theoremstyle{definition}
   \newtheorem{definition}[theorem]{Definition}
   \newtheorem{example}[theorem]{Example}
\theoremstyle{remark}    
  \newtheorem{remark}[theorem]{Remark}

\usepackage{graphicx}
\graphicspath{ {./Images/} }


\title{CPSC-354 Report}
\author{Sharon Chang  \\ Chapman University}

\date{\today}

\begin{document}

\maketitle

\begin{abstract}
Placeholder  
\end{abstract}

\tableofcontents

\section{Introduction}\label{intro}

Placeholder

\section{Homework}\label{homework}


\subsection{Week 1}

\begin{lstlisting}
def numInput():
    while True:
        numStr = input("Enter a number: ")
        try:
            num = int(numStr)
            if num > 0:
                return num
            else:
                print("The number must be greater than 0.")
                continue
        except:
            print("Invalid input.")

def gcd(numA, numB):
    if (numA > numB):
        result = gcd(numA-numB, numB)
    elif (numA < numB):
        result = gcd(numA, numB-numA)
    else:
        result = numA
    return result

# Main
numA = numInput()
print("First number acquired.")
numB = numInput()
print("Second number acquired.")

result = gcd(numA, numB)
print("The GCD of " + str(numA) + " and " + str(numB) + " is: " + str(result) + ".")
\end{lstlisting}
%
I chose to write my GCD program in Python.

\medskip\noindent
The numInput function first gets inputs from the user and makes sure that they are valid inputs to perform the gcd function on. The while loop traps the user until they input a valid non-zero integer. This repeats twice to obtain two numbers.

\medskip\noindent
After that, the two numbers are put into the gcd function. If number A is greater than number B, it recursively calls on the function again to find the GCD between the difference of A and B and number B. If number B is greater than number A, it recursively calls on the function again to find the GCD between number A and the difference of B and A. This repeats until the two numbers being compared are equal, after which the function returns the equal number. This equal number is the GCD of the two given numbers. This recursive function works because the method of subtraction will repeatedly lower the numbers until they inevitably equal out to yield a result.

\subsection{Week 2}

\begin{lstlisting}
select_evens [] = []
select_evens (x:xs) | length (x:xs) >= 2 = (head xs) : (select_evens (tail xs))
                    | length (x:xs) == 1 = []
                    | otherwise = x:xs

select_odds [] = []
select_odds (x:xs) | length (x:xs) >= 1 = x : (select_odds (tail xs))
                   | otherwise = x:xs

member i [] = False
member i (x:xs) | i == x = True
                | otherwise = member i xs

append [] [] = []
append xs [] = xs
append [] ys = ys
append xs ys = head xs : (append (tail xs) ys)

revert [] = []
revert xs = append (revert (tail xs)) (head xs : [])

less_equal [] [] = True
less_equal xs [] = False
less_equal [] ys = True
less_equal xs ys | head xs <= head ys = less_equal (tail xs) (tail ys)
                 | otherwise = False
\end{lstlisting}
%
For my select evens function, I set the base case to return an empty list. The recursive function has 3 cases. In the first case, provided the given list has a length of 2 or greater, the head of the list xs is taken as the first element of the list. This is the second element of the given list because x is the head of list x:xs. After that the tail of the list xs is put back into select evens, which removes the first two elements from the list x:xs. In the second case, if the length of the list is 1, it also returns an empty list. Otherwise, it just returns the given list.

\begin{lstlisting}
select_evens [1,2,3,4,5] =
	2 : (select_evens [3,4,5]) =
	2 : 4 : (select_evens [5]) =
	2 : 4 : [] =
	[2,4]
\end{lstlisting}
%
My select odds function is similar to my select evens function, with a base case of an empty list. If the length of the given list is greater than or equal to 1, the head of list x:xs, which is the value x, is taken as the first element of the list. The tail is then passed back into select odds for the same reasons as the select evens function. If the length is too short, it returns the given list.

\begin{lstlisting}
select_odds [1,2,3,4,5] =
	1 : (select_odds [3,4,5]) =
	1 : 3 : (select_odds [5]) =
	1 : 3 : 5 : [] =
	[1,3,5]
\end{lstlisting}
%
The member function returns a base case of False given any input with an empty list. When given a list, as long as the list is greater than or equal to 1, it checks to see if the given input value matches the head of the given list. If it matches, it returns true, otherwise, it passes the tail of the list back into the member function.

\begin{lstlisting}
member 3 [1,2,3,4,5] =
	member 3 [2,3,4,5] =
	member 3 [3,4,5] =
	True
\end{lstlisting}
%
When given an empty set of lists, the append function returns an empty list. If only one of the given lists is empty, it returns the non-empty list. When given two non-empty lists, the head of the first list is taken as the first element of the list. It is attached to the resulting list from appending the tail of the first list with the whole of the second list. Eventually, this empties out the first list, which will cause the entirety of the second list to be appended onto the elements of the first list.

\begin{lstlisting}
append [1,2,3] [4,5,6] =
	1 : (append [2,3] [4,5,6]) =
	1 : 2 : (append [3] [4,5,6]) =
	1 : 2 : 3 : (append [] [4,5,6]) =
	1 : 2 : 3 : [4,5,6] =
	[1,2,3,4,5,6]
\end{lstlisting}
%
The base case of the revert function returns an empty list when given an empty list. When given a non-empty list, the revert function uses the previous append function to create a new list where the head of the given list is placed at the back. The remaining list elements are passed back into the revert function, which will result in a list where the elements are in reversed order.

\begin{lstlisting}
revert [1,2,3,4,5] =
	(append (revert [2,3,4,5])) : 1 : [] =
	(append (revert [2,3,4,5])) : [1] =
	(append (revert [3,4,5])) : 2 : [] : [1] =
	(append (revert [3,4,5])) : [2] : [1] =
	(append (revert [4,5])) : 3 : [] : [2] : [1] =
	(append (revert [4,5])) : [3] : [2] : [1] =
	(append (revert [5])) : 4 : [] : [3] : [2] : [1] =
	(append (revert [5])) : [4] : [3] : [2] : [1] =
	(append (revert [])) : 5 : [] : [4] : [3] : [2] : [1] =
	(append (revert [])) : [5] : [4] : [3] : [2] : [1] =
	[] : [5] : [4] : [3] : [2] : [1] =
	[5,4,3,2,1]
\end{lstlisting}
%
When the two lists given to the less equal function are both empty, the function returns a true. If the first list is non-empty but the second list is, the function returns false. If the second list is non-empty but the first one is, the function returns true. If both lists are non-empty lists, the head of both lists is compared to one another. If the head of the first list is less than or equal to the second list, the function continues by passing on the tails of both lists back into itself. Otherwise, if the head of the first list is greater than the head of the second list, the function stops and returns a false.

\begin{lstlisting}
less_equal [1,1,4] [1,2,3] =
	less_equal [1,4] [2,3] =
	less_equal [4] [3] =
	False
\end{lstlisting}

\subsection{Week 3}
\begin{lstlisting}
hanoi 5 0 2
	hanoi 4 0 1
		hanoi 3 0 2
			hanoi 2 0 1
				hanoi 1 0 2 = move 0 2
				move 0 1
				hanoi 1 2 1 = move 2 1
			move 0 2
			hanoi 2 1 2
				hanoi 1 1 0 = move 1 0
				move 1 2
				hanoi 1 0 2 = move 0 2
		move 0 1
		hanoi 3 2 1
			hanoi 2 2 0
				hanoi 1 2 1 = move 2 1
				move 2 0
				hanoi 1 1 0 = move 1 0
			move 2 1
			hanoi 2 0 1
				hanoi 1 0 2 = move 0 2
				move 0 1
				hanoi 1 2 1 = move 2 1
	move 0 2
	hanoi 4 1 2
		hanoi 3 1 0
			hanoi 2 1 2
				hanoi 1 1 0 = move 1 0
				move 1 2
				hanoi 1 0 2 = move 0 2
			move 1 0
			hanoi 2 2 0
				hanoi 1 2 1 = move 2 1
				move 2 0
				hanoi 1 1 0 = move 1 0
		move 1 2
		hanoi 3 0 2
			hanoi 2 0 1
				hanoi 1 0 2 = move 0 2
				move 0 1
				hanoi 1 2 1 = move 2 1
			move 0 2
			hanoi 2 1 2
				hanoi 1 1 0 = move 1 0
				move 1 2
				hanoi 1 0 2 = move 0 2
\end{lstlisting}
%
These instructions count the leftmost pole as the Pole 1, the middle pole as Pole 2, and the rightmost pole as Pole 3.

\medskip\noindent
Move the top disk from Pole 1 to Pole 3. Next, move the next top disk from Pole 1 to Pole 2. Take the top disk from Pole 3 and move it to Pole 2. Then, move the top disk from Pole 1 to Pole 3. Follow by moving the top from Pole 2 to Pole 1. Then, move the top disk from Pole 2 to Pole 3. Afterwards, move the top disk from Pole 1 to Pole 3. Move the next top disk from Pole 1 to Pole 2. Next, take the top disk from Pole 3 and move it to Pole 2. Take the next top disk from Pole 3 and move it to Pole 1. Then, take the top disk from Pole 2 and move it to Pole 1. Follow by moving the top disk from Pole 3 to Pole 2. Next, move the top disk from Pole 1 to Pole 3. Then, move the next top disk from Pole 1 to Pole 2. Return to the top disk of Pole 3 and move it to Pole 2. Now, take the top disk of Pole 1 and move it to Pole 3. Next, take the top disk from Pole 2 and move it to Pole 1. Then, take the top disk from Pole 2 and move it to Pole 3. Follow by taking the top disk from Pole 1 and moving it to Pole 3. Return to Pole 2 and move its top disk to Pole 1. Now, take the top disk from Pole 3 and move it to Pole 2. Next, take the top disk from Pole 3 and move it to Pole 1. Then, take the top disk from Pole 2 and move it to Pole 1. Take the next top disk from Pole 2 and move it to Pole 3. Now, take the top disk from Pole 1 and move it to Pole 3. Return to the next top disk from Pole 1 and move it to Pole 2. Then, take the top disk from Pole 3 and move it to Pole 2. Next, take the top disk from Pole 1 and move it to Pole 3. Then, take the top disk from Pole 2 and move it to Pole 1. Move the next top disk from Pole 2 to Pole 3. Then, take the top disk from Pole 1 and move it to Pole 3. A 5-disk Tower of Hanoi has now been successfully moved from Pole 1 to Pole 3.

\medskip\noindent
The word hanoi appears 31 times in the computation.

\medskip\noindent
\begin{align*}
{\rm hanoi} (1) (x) (y)
& = {\rm move}(x)(y)
\end{align*}
\begin{align*}
{\rm hanoi} (n+1) (x) (y)
& = {\rm hanoi} (n) (x) ({\rm other}(x)(y)) \\
& = {\rm move}(x)(y)\\
& = {\rm hanoi} (n) ({\rm other}(x)(y)) (y)
\end{align*}

\subsection{Week 4}
\includegraphics {hw4no1}

\includegraphics {hw4no2}

\includegraphics {hw4no3}

\includegraphics {hw4no4}

\includegraphics {hw4no5}

\subsection{Week 5}
\begin{lstlisting}
x
\end{lstlisting}
%
\includegraphics {hw5no1}

\begin{lstlisting}
x y
\end{lstlisting}
%
\includegraphics {hw5no2}

\begin{lstlisting}
x y z
\end{lstlisting}
%
\includegraphics {hw5no3}

\begin{lstlisting}
\ x . x
\end{lstlisting}
%
\includegraphics {hw5no4}

\begin{lstlisting}
\ x . x x
\end{lstlisting}
%
\includegraphics {hw5no5}

\begin{lstlisting}
\ x . \ y . x y (\ x . x) z
\end{lstlisting}
%
\includegraphics {hw5no6}

\begin{lstlisting}
\ x . \ y . x y z a b c
\end{lstlisting}
%
\includegraphics {hw5no7}


\begin{lstlisting}
(\x.x) a
a
\end{lstlisting}
%

\begin{lstlisting}
\x.x a
a
\end{lstlisting}
%

\begin{lstlisting}
(\x.\y.x) a b
(\y) a b
(\y a) b
\y a b
\end{lstlisting}
%

\begin{lstlisting}
(\x.\y.y) a b
(\x) a b
(\x a) b
\x a b
\end{lstlisting}
%

\begin{lstlisting}
(\x.\y.x) a b c
(\y) a b c
(\y a) b c
(\y a b) c
\y a b c
\end{lstlisting}
%

\begin{lstlisting}
(\x.\y.y) a b c
(\x) a b c
(\x a) b c
(\x a b) c
\x a b c
\end{lstlisting}
%

\begin{lstlisting}
(\x.\y.x) a (b c)
(\y) a (b c)
(\y a) (b c)
\y a b c
\end{lstlisting}
%

\begin{lstlisting}
(\x.\y.y) a (b c)
(\x) a (b c)
(\x a) (b c)
\x a b c
\end{lstlisting}
%

\begin{lstlisting}
(\x.\y.x) (a b) c
(\y) (a b) c
(\y a b) c
\y a b c
\end{lstlisting}
%

\begin{lstlisting}
(\x.\y.y) (a b) c
(\x) (a b) c
(\x a b) c
\x a b c
\end{lstlisting}
%

\begin{lstlisting}
(\x.\y.x) (a b c)
(\y) (a b c)
\y (a b c)
\y a b c
\end{lstlisting}
%

\begin{lstlisting}
(\x.\y.y) (a b c)
(\x) (a b c)
\x (a b c)
\x a b c
\end{lstlisting}
%

\begin{lstlisting}
evalCBN (EApp (\x.x) ((\y.y)a))
   (\x.x) (a)
   (a)
\end{lstlisting}
%

\subsection{Week 6}
\begin{lstlisting}
(\exp . \two . \three . exp two three)
(\m.\n. m n)
(\f.\x. f (f x))
(\f.\x. f (f (f x)))
=
((\m.\n. m n) (\f.\x. f (f x)) (\f2.\x2. f2 (f2 (f2 x2))))
=
((\n. (\f.\x. f (f x)) n) (\f2.\x2. f2 (f2 (f2 x2))))
=
(((\f.\x. f (f x)) (\f2.\x2. f2 (f2 (f2 x2)))))
=
(((\x. (\f2.\x2. f2 (f2 (f2 x2))) ((\f2.\x2. f2 (f2 (f2 x2))) x))))
=
(((\x. (\x2. ((\f2.\x2. f2 (f2 (f2 x2))) x) (((\f2.\x2. f2 (f2 (f2 x2))) x) (((\f2.\x2. f2 (f2 (f2 x2))) x) x2))))))
=
(((\x. (\x2. ((\x2. x (x (x x2)))) (((\f2.\x2. f2 (f2 (f2 x2))) x) (((\f2.\x2. f2 (f2 (f2 x2))) x) x2))))))
=
(((\x. (\x2. (\x2. x (x (x x2))) (((\f2.\x2. f2 (f2 (f2 x2))) x) (((\f2.\x2. f2 (f2 (f2 x2))) x) x2))))))
=
(((\x. (\x2. (x (x (x (((\f2.\x2. f2 (f2 (f2 x2))) x) (((\f2.\x2. f2 (f2 (f2 x2))) x) x2)))))))))
=
(((\x. (\x2. (x (x (x (((\x2. x (x (x x2)))) (((\f2.\x2. f2 (f2 (f2 x2))) x) x2)))))))))
=
(((\x. (\x2. (x (x (x ((\x2. x (x (x x2))) (((\f2.\x2. f2 (f2 (f2 x2))) x) x2)))))))))
=
(((\x. (\x2. (x (x (x ((x (x (x (((\f2.\x2. f2 (f2 (f2 x2))) x) x2))))))))))))
=
(((\x. (\x2. (x (x (x ((x (x (x (((\x2. x (x (x x2))) x) x2))))))))))))
=
(((\x. (\x2. (x (x (x ((x (x (x (((x (x (x x)))) x2))))))))))))
=
((\x. (\x2. (x (x (x (x (x (x ((x (x (x x))) x2))))))))))
\end{lstlisting}

\section{Project}

Placeholder

\subsection{Specification}

Placeholder

\subsection{Prototype}

Placeholder

\subsection{Documentation}

Placeholder

\subsection{Critical Appraisal}

Placeholder

\section{Conclusions}\label{conclusions}

Placeholder

\begin{thebibliography}{99}
\bibitem[PL]{PL} \href{https://github.com/alexhkurz/programming-languages-2022/blob/main/README.md}{Programming Languages 2022}, Chapman University, 2022.
\end{thebibliography}

\end{document}